\chapter{Introduction}

    \section{Principe de base}

        Pour notre projet, nous avons décidé d'adapter le jeu de plateau Cthulhu Wars
        sous forme d'un jeu informatique (Anime Warfare), dans un tout autre univers :
        celui de monde de l'animation. Le but de ce jeu est simple : être le premier à
        réaliser ses six productions majeures, tout en obtenant un maximum de fans.
        Chaque joueur joue chacun son tour durant diverses phases, et peut obtenir
        des capacités uniques, liés à sa license (son ''équipe'').
        \newline
        Chaque joueur dispose d'unités qu'il peut, moyennant un coût, placer sur une
        carte ; les unités ainsi placée s'affrontent alors dans des batailles d'insultes
        sanglantes, dans le but d'humilier leurs adversaires à mort.

    \section{Principales différences avec Cthulhu Wars}

        L'univers étant différent de celui de base, les termes ont étés adaptés :
        \begin{center}
            \begin{tabular}{|l|l|}
                \hline
                \textbf{Cthulhu Wars} & \textbf{Anime-Warfare} \\
                \hline
                \hline
                Factions & Licenses \\
                \hline
                Grand Ancien & Héro \\
                \hline
                Unités & Personnages \\
                \hline
                Adorateurs & Mascottes \\
                \hline
                Portail d'invocation & Studio \\
                \hline
                Grimoires & Productions majeures \\
                \hline
                Points de pouvoirs & Employés du staff \\
                \hline
                Rituels de l'apocalypse & Conventions \\
                \hline
                Phase de l'apocalypse & Phase marketing \\
                \hline
                Échelle de l'apocalypse & Piste marketing \\
                \hline
                Signes des anciens & Droits de campagnes publicitaires \\
                \hline
            \end{tabular}
        \end{center}

        Le style de jeu reste principalement le même, mais on notera toutefois une
        différence importante : l'apparition d'un deck global. \\
        Le premier joueur d'un tour devra parfois tirer une carte dans ce deck, en
        choisissant parmi ses trois premières cartes. Un effet va alors s'appliquer
        pendant un nombre donné de tours à tout les joueurs, influançant le déroulement
        de la partie.
